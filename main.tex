\documentclass[12pt, a4paper]{article}
\usepackage[utf8]{inputenc}
\usepackage[brazil, english]{babel}
\usepackage[T1]{fontenc}
\usepackage{amsmath}
\usepackage{setspace}
\usepackage{indentfirst}
\usepackage{natbib}
\usepackage[left=3.5cm,top=3cm,right=3cm,bottom=3cm]{geometry}
\usepackage{lineno}

\bibliographystyle{apalike2}

\title{\textbf{Testing the Assignment of Cases in Brazilian Supreme Court: a Bayesian framework}}
\author{Guilherme Jardim Duarte  \and Julio Canello}

\begin{document}
\onehalfspacing
\maketitle

\begin{abstract}

How does it work the process of assignment of cases and actions to justices in Brazilian Supreme Court? In this paper, we propose a model for testing the random assignment of cases, employing a Bayesian approach. This model can be extended to other Courts in Brazil. We show that ...




\vspace{.5cm}
\noindent
\textsc{Keywords}: Judicial Politics, Brazilian Supreme Court, Assignment of Judicial Cases
\end{abstract}

\newpage

\linenumbers
\section{Introduction}


\section{Data}


We gather data from the site of the Brazilian Supreme Court.

\section{Model}




\section{Results}

\bibliography{biblio}

\end{document}